% Chapter Template

\chapter{Introduction and Background} % Main chapter title

\label{Chapter1} % Change X to a consecutive number; for referencing this chapter elsewhere, use \ref{ChapterX}

\lhead{Chapter 1. \emph{Introduction and Background}} % Change X to a consecutive number; this is for the header on each page - perhaps a shortened title

%----------------------------------------------------------------------------------------
%	SECTION 1
%----------------------------------------------------------------------------------------

\section{Normoxia and Hypoxia in Cells and tissues}

Oxygen makes up approximately 21\% of the atmospheric air that enters the lung, yet most tissues of the body experience significantly less.
Normoxia is the range of oxygen levels that tissues experience in a healthy system.
Within the body oxygen levels vary depending on the specific tissue.
Oxygen levels in lung alveoli are ~13\%, yet different regions of the brain can range from 7\% to less than 1\% (A list of reported oxygen levels can be found in Table oxygen in tissues).
Despite the fact that normoxia for most tissues is far below 21\%, culture studies are typically done at atmospheric levels. This is in effect hyperoxic, which is above physiological oxygen conditions.

%-----------------------------------
%	SUBSECTION 1
%-----------------------------------
\subsection{Hypoxia inducible factor}

Tissue oxygenation is a constant balancing act between oxygen availability and oxygen consumption.
Small changes in oxygen levels trigger homeostasis responses in cells and tissues.
Research has found that hypoxia induces extensive alterations in gene expression.
The hypoxia inducible factor (HIF) family of transcription factors have been shown to activate as oxygen levels fall and the work to regulate gene expression that promotes glycolysis, angiogenesis.
 

%----------------------------------------------------------------------------------------
%	SECTION 2
%----------------------------------------------------------------------------------------

\section{Measurement of oxygen}

To investigate the effects of oxygen on biological systems one first needs to find a suitable method for measuring it. Measuring oxygen in the cell environment is even more challenging if temporal and spatial requirements are strict. Stoichiometric methods such as the Winkler method \cite{winkler1888} are too slow and require fixing of samples.

\subsection{Clark-style electrodes}

Clark-style electrodes \cite{Clark1953} typically use a platinum working electrode and silver chloride reference electrode with potassium chloride for the electrolyte.
A voltage of about 800 mV, which is sufficient to reduce oxygen, is applied across the electrodes.
Oxygen is reduced at the working electrode, thus producing electrons or current proportional to the amount of oxygen present.
The electrodes and electrolyte are protected behind a gas-permeable layer of polytetrafluoroethylene (PTFE) to prevent adsorption of proteins or interfering ions from fouling the electrodes.
Because Clark-style electrodes consume oxygen in order to detect it, stirring of the sample is usually required for fast response measurements.
The electrode is also very sensitive to changes in sample temperature.
Clark electrodes are unreliable for long-term measurements for a number of reasons which contribute to unstable readings: depletion of the electrolyte, the production of \ce{OH-} ions affecting the pH causing zero drift, and the anode becoming coated in \ce{AgCl}.
If used with biological samples, the protective, PTFE membrane will also lose permeability over time due to the adsorption of protein and other residues.
Clark-style electrodes also suffer from low temporal and spatial resolution due to the time it takes for oxygen to diffuse across the PTFE membrane and to the electrodes.
In addition, the relative size of these probes (~3+ mm diameter probe) makes interfacing with microfluidic channels problematic.

\subsection{Optical Oxygen Sensors}

Formicrofluidic systems, optical oxygen sensors are the tool of choice. They have several advantages overClark-style electrodes. They do not consume oxygen so they can be used in low or no flow environments and do not suffer from fouling, making them stable for long-term studies. Where Clark electrodes require an electrical connection to each position to bemeasured and only provide a single, low spatial resolutionmeasurement, optical sensors allowmeasurement over the entire area of the sensor and at any number of discrete points. These sensors take advantage of oxygen-indicating fluoro-
phores that are quenched in the presence of oxygen. The degree of quenching is determined by the oxygen partial pressure. The relationship between intensity and oxygen partial pres- sure is described by the Stern–Volmer equation:

equation



\section{Methods for Global Oxygen Control and Measurement}



\subsection{Subsection 1}



\section{Microfluidic Methods for Oxygen Control}


%----------------------------------------------------------------------------------------
%	SECTION 3
%----------------------------------------------------------------------------------------


\section{Horizontal Gene Transfer}

\subsection{History and discoveries}
In 1928 Griffith reported on the discovery of the `transforming principal' where mice that were injected with a non-virulent strain resulted in a lethal infection when injected along with a heat killed virulent strain\cite[Griffith1928].
Both virulent and non-virlent strains could then be isolated from the blood of the dead mouse.
Somehow the presence of the virulent strain, although dead, passed information that allowed the non-virulent strain to transform and become virulent.
A subsequent study by Avery, McLeod and McCarty demonstrated that the presence of an overlooked substance, DNA, was responsible for re-programing strains\cite[Avery1944].
At the time it was believed that some yet to be discovered protein complex was the carrier of genetic information, but after this result scientists began looking more closely at DNA.
In 1951 Freeman demonstrated that a phage could re-program \textit{Corynebacterium diphtheriae} from non-virulent to virulent\cite{Freeman1951}.
Hershey and Chase found that the DNA is incorporated into the cell rather than the protein from the phage\cite{Hershey1952}.
They confirmed this by producing two groups of phages, one with a radio labeled protein and one with radio labeled DNA.
Watson, one of the scientist that discovered the structure of DNA was a phage scientist as well.

\subsection{Mechanisms}

Horizontal gene transfer (HGT) or (sometimes lateral gene transfer) describes the introduction of genes from an outside source which is distinct from vertical gene transfer where genes are passed from mother to daughter cells. The mechanisms of HGT are transformation, transduction and conjugation.



\textbf{Transduction} involves a phage which carries and infects the host cell by injecting the DNA or RNA into the cytoplasm. 

In \textbf{conjugation} genetic material is passed directly during cell-to-cell contact.

\subsection{Transformation}

\textbf{Transformation} is the uptake of exogenous DNA from the environment by a cell\cite{Johnston2014}.
Many species naturally transformable species of bacteria have been reported such as \textit{Streptococcus pneumoniae},  
Species such as \textit{Streptococcus pneumoniae} undergo transformation naturally.
Transformation was first described in \textit{Streptococcus pneumoniae} \cite{Griffith1928}.
In transformation the recipient cell controls the process and expresses all the proteins required for DNA up-take and homologous recombination.

\subsection{Inducible Competence}

\textit{Streptococcus pneumoniae} relies on quorum sensing to activates to a state call competence.
Upon reaching a density of xxxx cells release a peptide that induces competence.
This peptide has been aptly named competence stimulating peptide.

\subsection{Competence Induced Fratricide}
 
\subsection{Transformation in Vitro Vs in Vivo}

\subsection{Global Heath Considerations}

\section{Droplet Microfluidics}

\subsection{cell encapsulation}

\subsection{Poisson distribution}





