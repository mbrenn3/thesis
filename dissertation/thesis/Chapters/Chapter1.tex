% Chapter Template

\chapter{Introduction and Background} % Main chapter title

\label{Chapter1} % Change X to a consecutive number; for referencing this chapter elsewhere, use \ref{ChapterX}

\lhead{Chapter 1. \emph{Introduction and Background}} % Change X to a consecutive number; this is for the header on each page - perhaps a shortened title

%----------------------------------------------------------------------------------------
%	SECTION 1
%----------------------------------------------------------------------------------------

\section{Main Section 1}

Lorem ipsum dolor sit amet, consectetur adipiscing elit. Aliquam ultricies lacinia euismod. Nam tempus risus in dolor rhoncus in interdum enim tincidunt. Donec vel nunc neque. In condimentum ullamcorper quam non consequat. Fusce sagittis tempor feugiat. Fusce magna erat, molestie eu convallis ut, tempus sed arcu. Quisque molestie, ante a tincidunt ullamcorper, sapien enim dignissim lacus, in semper nibh erat lobortis purus. Integer dapibus ligula ac risus convallis pellentesque.

%-----------------------------------
%	SUBSECTION 1
%-----------------------------------
\subsection{Subsection 1}

Nunc posuere quam at lectus tristique eu ultrices augue venenatis. Vestibulum ante ipsum primis in faucibus orci luctus et ultrices posuere cubilia Curae; Aliquam erat volutpat. Vivamus sodales tortor eget quam adipiscing in vulputate ante ullamcorper. Sed eros ante, lacinia et sollicitudin et, aliquam sit amet augue. In hac habitasse platea dictumst.

%-----------------------------------
%	SUBSECTION 2
%-----------------------------------

\subsection{Subsection 2}
Morbi rutrum odio eget arcu adipiscing sodales. Aenean et purus a est pulvinar pellentesque. Cras in elit neque, quis varius elit. Phasellus fringilla, nibh eu tempus venenatis, dolor elit posuere quam, quis adipiscing urna leo nec orci. Sed nec nulla auctor odio aliquet consequat. Ut nec nulla in ante ullamcorper aliquam at sed dolor. Phasellus fermentum magna in augue gravida cursus. Cras sed pretium lorem. Pellentesque eget ornare odio. Proin accumsan, massa viverra cursus pharetra, ipsum nisi lobortis velit, a malesuada dolor lorem eu neque.

%----------------------------------------------------------------------------------------
%	SECTION 2
%----------------------------------------------------------------------------------------

\section{Main Section 2}

\subsection{Subsection 1}
Sed ullamcorper quam eu nisl interdum at interdum enim egestas. Aliquam placerat justo sed lectus lobortis ut porta nisl porttitor. Vestibulum mi dolor, lacinia molestie gravida at, tempus vitae ligula. Donec eget quam sapien, in viverra eros. Donec pellentesque justo a massa fringilla non vestibulum metus vestibulum. Vestibulum in orci quis felis tempor lacinia. Vivamus ornare ultrices facilisis. Ut hendrerit volutpat vulputate. Morbi condimentum venenatis augue, id porta ipsum vulputate in. Curabitur luctus tempus justo. Vestibulum risus lectus, adipiscing nec condimentum quis, condimentum nec nisl. Aliquam dictum sagittis velit sed iaculis. Morbi tristique augue sit amet nulla pulvinar id facilisis ligula mollis. Nam elit libero, tincidunt ut aliquam at, molestie in quam. Aenean rhoncus vehicula hendrerit.


%----------------------------------------------------------------------------------------
%	SECTION 3
%----------------------------------------------------------------------------------------


\section{Horizontal Gene Transfer}

\subsection{History and discoveries}
In 1928 Griffith reported on the discovery of the `transforming principal' where mice that were injected with a non-virulent strain resulted in a lethal infection when injected along with a heat killed virulent strain\cite[Griffith1928].
Both virulent and non-virlent strains could then be isolated from the blood of the dead mouse.
Somehow the presence of the virulent strain, although dead, passed information that allowed the non-virulent strain to transform and become virulent.
A subsequent study by Avery, McLeod and McCarty demonstrated that the presence of an overlooked substance, DNA, was responsible for re-programing strains\cite[Avery1944].
At the time it was believed that some yet to be discovered protein complex was the carrier of genetic information, but after this result scientists began looking more closely at DNA.
In 1951 Freeman demonstrated that a phage could re-program \textit{Corynebacterium diphtheriae} from non-virulent to virulent\cite{Freeman1951}.
Hershey and Chase found that the DNA is incorporated into the cell rather than the protein from the phage\cite{Hershey1952}.
They confirmed this by producing two groups of phages, one with a radio labeled protein and one with radio labeled DNA.
Watson, one of the scientist that discovered the structure of DNA was a phage scientist as well.

\subsection{Mechanisms}

Horizontal gene transfer (HGT) or (sometimes lateral gene transfer) describes the introduction of genes from an outside source which is distinct from vertical gene transfer where genes are passed from mother to daughter cells. The mechanisms of HGT are transformation, transduction and conjugation.



\textbf{Transduction} involves a phage which carries and infects the host cell by injecting the DNA or RNA into the cytoplasm. 

In \textbf{conjugation} genetic material is passed directly during cell-to-cell contact.

\subsection{Transformation}

\textbf{Transformation} is the uptake of exogenous DNA from the environment by a cell\cite{Johnston2014}.
Many species naturally transformable species of bacteria have been reported such as \textit{Streptococcus pneumoniae},  
Species such as \textit{Streptococcus pneumoniae} undergo transformation naturally.
Transformation was first described in \textit{Streptococcus pneumoniae} \cite{Griffith1928}.
In transformation the recipient cell controls the process and expresses all the proteins required


