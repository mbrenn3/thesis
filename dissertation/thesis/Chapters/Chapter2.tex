% Chapter Template

\chapter{Materials and Methods} % Main chapter title

\label{Chapter2} % Change X to a consecutive number; for referencing this chapter elsewhere, use \ref{ChapterX}

\lhead{Chapter 2. \emph{Materials and Methods}} % Change X to a consecutive number; this is for the header on each page - perhaps a shortened title

%----------------------------------------------------------------------------------------
%	SECTION 1
%----------------------------------------------------------------------------------------

\section{3D Printed Microfluidic Devices}

\subsection{Fabrication of the 3D-printed Part}
The fluidic distribution networks were designed in Autocad in 2D and then extruded to 3D with Blender.
Because the channel path length to each well varied the flow resistance in each channel was balanced by adjusting the width.
Each distribution network services six pillars that reach into the well of the plate leaving a 500 $\mu$m gap to the culture surface.
A pipe within a pipe design was used to allow a forward and return path in each pillar that also created uniform flow across the diffusion membrane.
The design was repeated to create four sets that service six wells.
Hose barbs were added to each inlet and outlet port.
The resulting STL file was printed by fineline prototyping with a Viper SLA system in WaterShed XC, an ABS or PBT-like proprietary material.

\subsection{Attachment of Membranes}
Gas permeable membranes were made by compressing 10:1 mixed and degassed PDMS between two glass plates that were spaced 100 $\mu$m apart with scotch tape.
Molds were baked at 50$^{\circ}$C on a hot plate to avoid bubble formation.
Membranes were transfered to a transparency and placed over a cutting template and cut to size for the pillar bottoms.
To attach the membranes to the 3D printed part, a small amount of PDMS was applied to the membranes and spread thin to act as a mortar.
The membranes were left in place to cure overnight.

\subsection{Flow set-up}
Flow though the device was driven by vacuum with a ventri sink aspirator.
This vacuum line was split evenly into four and connected to each outlet.
The inlet gas was from pressurized tanks of either 0\% \ce{O2}, 5\% \ce{O2}, 7.5\% \ce{O2}, or 21\% \ce{O2}.
Each gas tank was also 5\% \ce{CO2} for \ce{CO2} buffered medium.
Each pressurized tank had a multistage cylinder regulator as well as an in-line IV flow regulator.
An open connection was made in the lines leading to the inlet by placing tubing from the IV regulator into the wide end of a 1000 $\mu$L pipette tip.
The narrow end of the pipette tip was connected to tubing from the inlet barbs of the insert.
Flow from the pressurized tank was set to exceed the flow of the vacuum so that gas was drawn from the tank condition.
Flow allowed by the IV flowmeter in each of the four inlet lines was measured by temporarily diverting the flow with a three-way stopcock to a floating ball rotameter(Figure (IV flow meters)).

\subsection{Oxygen Characterization}
PtOEPK (Pt(II) Octaethylporphine ketone) sensors were created by spinning thin films from a PtOEPK in polystyrene-toluene solution.
Thin film sensors were cut and fixed to the bottom of each well of a 24-well with PDMS.
An oxygen scavenger solution was made by dissolving \ce{Na2SO3} in DI water at 2 g/L.
The wells were filled with the oxygen scavenger solution and the insert device was placed in the plate.
5\% \ce{CO2}, balanced \ce{N2} was flowed through the device to flush out any oxygen.
The intensity of the PtOEPK sensor was measured via epi-fluorescence imaging (Olympus XI-51).
A filter cube of with ex/em was used.
When the intensity stabilized images of all the sensors were taken.
A motorized stage and imaging control software (metamorph) was used to memorize positions of all the wells.
Each well was imaged sequentially in a 1-6, 12-7, 13-18, 24-19, order to minimize the distance and time the stage had to travel.
These intensity values were considered 0\% oxygen and taken as the calibration point.
The plate was removed from the microscope and the insert was taken out and both the plate and insert were rinsed in DI water to wash away all the scavenger solution.
DI water that was equilibrated to room temperature and atmospheric air (by placing in an open container overnight in the microscope room) was used to fill the wells.
With the insert placed in the plate and 5\% \ce{CO2}, balanced \ce{N2} was flowed through the device until the intensity stabilized (~45 minutes).
The 5\% \ce{CO2}, 5\% \ce{O2}, balanced \ce{N2} was flowed through the device.
When the intensity stabilized a calibration point was taken that was considered 5\% \ce{O2}.
Next 5\% \ce{CO2}, 10\% \ce{O2}, balanced \ce{N2} was flowed through the device and the 10\% calibration point was taken.
Finally the water in the plate was replaced again and 5\% \ce{CO2}, balanced air was flowed through the device.

\textbf{Timecourse Oxygen Characterization Data Collection:} 
Each well was imaged sequentially at intervals of 5 minutes for a total time of 6 hours.
After 15 minutes of 5\% \ce{CO2}, balanced air (Three time points), the inlets were changed to: row 1: 5\% \ce{CO2} balanced \ce{N2}; row 2: 5\% \ce{CO2}, 5\% \ce{O2} balanced \ce{N2}; row 3: 5\% \ce{O2}, 10\% \ce{O2}, balance \ce{N2}; row 4: 5\% \ce{CO2} balanced air.
Summarized in table (Oxygen characterization rows)

\subsection{Analyzing Intensities}
ImageJ was used to organize images according to meta data into folders of wells including specifying calibration points.
Then imageJ was used to measure intensity of each image from the time course data which was recorded into a text file for each well (Imagej macros can be found in appendix (code)).
A two-site Stern-Volmer model was used to interpret the intensity data and generate oxygen values (Matlab code can be found in Appendix (code)).




    

%----------------------------------------------------------------------------------------
%	SECTION 2
%----------------------------------------------------------------------------------------

\section{Microbiology}

\subsection{\textit{Streptococcus pneumoniae} strains}

Two strains of \textit{Streptococcus pneumoniae} were produced by Dr. Morrison of the Biological Sciences Department at UIC: CP2204 and CP2215.
The strains were made to be complementary in antibiotic resistances to aid in assaying transformation events.
The CP2204 strain has inducible competence so is the designated recipient where CP2215 can not become competent so it is the natural donor of DNA.
Traits of these strains are listed in table \ref{table_strains}

\begin{table}[]
	\centering
	\caption{Summary of phenotypes of strains used for transformation studies}
	\label{table_strains}
	\begin{tabular}{llllll}
		Strain & Competence    & Rifampin  & Spectinomycin & Novobiocin & Label \\
		CP2204 & CSP inducible & Resistant & Sensitive     & Sensitive  & RFP   \\
		CP2215 & non-competent & Sensitive & Resistant     & Resistant  & GFP  
	\end{tabular}
\end{table}

\subsection{Culturing}

\textit{Streptococcus pneumoniae} strains CP2204 and CP2215 were grown separately in 12 mL of CDM with 1\% CAT medium at 37$^{\circ}$C to the desired OD (media formulas can be found in Appendix \ref{AppendixA}).
When the desired OD was reached suspended cultures were spun down at 8000 RCF for 8 minutes in a 4$^{\circ}$C centrifuge.
The supernatant was poured off and the cultures were re-suspended in M9 or a ratio of up to 20\% CDM in M9 medium depending on the application and held at 4$^{\circ}$C.

\subsection{Inducing Transformation}

Strain CP2204 was made competent with and inducer cocktail containing CSP, BSA, and \ce{CaCl2} (The specific formula can be found in Apendeix \ref{AppendixA} Table \ref{table_csp}).
The inducer cocktail is introduced to the chilled mixture of CP2204 and CP2215.
The cells can be held in at 4$^{\circ}$C in the presence of the inducer cocktail and will remain inactivated.
This suspension is then brought up to 37$^{\circ}$C to initiate cell-cell attack and transformation.
Typically these reaction suspensions are transfered to a cryogenic vials and placed in a heater block at 37$^{\circ}$C for 30 minutes.
After 30 minutes suspension are diluted ten-fold into CAT media and incubated at 37$^{\circ}$C for an hour.
This hour incubation step dilutes the inducer inactivating it and allows the generation of recombinants to emerge.

\subsection{Plating and assaying for Drug Resistance}

Suspensions were diluted in more CAT media for anticipated survivor counts.
Typically suspensions were diluted by 100,000 for single drug, 100 for two drug agar, and 10 for triple drug agar.
Agar was made from either CAT medium or THY medium by adding 4.5 g of agar per 300 mL of medium and autoclaving.
50 mm plates are filled with the following layers in order: 

\begin{enumerate}
\item 3 mL Agar
\item 1.5 mL Agar + 75 $\mu$m to 1.5 mL cell suspension quickly mixed
\item 3 mL Agar
\item 3 mL drug agar
\end{enumerate}

Layers are added at least a few minutes apart to allow them to set before the next layer is added.
Plates are incubated at in a 37$^{\circ}$C room for 48 hours before counting.
Plates are incubated upside down to slow the diffusion of drug to the cell layer.
Concentrations in the drug layer are listed in Appendix \ref{AppendixA} Table \ref{table_drug-agar}.

\section{Droplet Generation and Cell Encapsulation}

A microfluidic droplet generating device was used to encapsulate cells during transformation.
A flow focusing device with two aqueous inlets was used to keep the cells and inducer separate until right before droplet generation.

\subsection{Fabrication of Droplet Generating Device}

\textbf{Master mold fabrication.} A 2D design was made in AutoCad and exported as a DFX format.
Small features such as as the filter arrays were saved in a separate file and then subtracted via boolean operation in LinkCAD.
SU-8 2015 (MicroChem Corp.) was spun at 4250 RPM on a 100 mm silicon wafer to a thickness of 15 $\mu$.
The wafer was baked at 95$^{\circ}$C for 10 minutes shielded from light.
The wafer was loaded into the $\mu$PG 101 (Heilderberg Instruments).
The converted file from LinkCAD was imported into the $\mu$PG 101 software.
The wafer was exposed at 18 mW, 100\% intensity and 4x exposure time.
After exposure the wafer was baked at 95$^{\circ}$C for 10 minutes and then at 120$^{\circ}$C for 30 minutes.
I found that this additional bake at 120$^{\circ}$C prevented de-lamination that is typical for SU-8 heights less than 50 $\mu$m.
The uncured SU-8 was then washed away with SU-8 developer (MicroChem Corp.) on a shaker for 30 minutes.
The wafer was then rinsed with acetone followed by IPA and blown dry with nitrogen.

\textbf{Casting, assembly and treatment of device.} About 30 grams of 10:1 monomer : curing-agent, PDMS (Slygard 184, Dow Corning) was mixed with a Thinky mixer for 2 minutes.
About 10 grams of mixed PDMS was poured onto the master mold and spun at 2000 RPM.
A 22 x 50 mm glass cover slip was layed on the uncured PDMS.
The mold was placed in a vacuum desiccator to pull air bubbles from under the cover slip.
The cover slip was then adjusted to be over the incubation chamber of the design and baked at 65$^{\circ}$C to set in place.
The remaining PDMS, ~20 grams, is poured on the mold and degassed.
The mold is then baked at 65$^{\circ}$C for 2 hours.
The PDMS is then carefully pealed from the mold containing the embedded cover slip.
Holes are punched with a 1 mm biopsy punch and the PDMS is placed in contact with a glass slide after 45 seconds of oxygen plasma exposure (Plasma Etch, Inc.) to permanently bond.
The completed chip is then baked at 135$^{\circ}$C for at least 1 hour to strengthen plasma bonding.
The chip is then removed and allowed to cool before treating with hydrophobic coating.
Novec 1720 (3M) is flowed through the device and then left to stand for 10 minutes.
The Novec is then flushed from the chip with air and returned to a 135$^{\circ}$C hot plate to bake in the coating and sterilize the chip for at least 30 minutes.

\subsection{Droplet generation and Cell Encapsulation}

The droplet generation chip uses a flow focusing design to make water-in-oil droplets (w/o).
The Oil and aqueous channels converge at a narrow neck where the aqueous-oil interface is focused causing droplets to pinch off in a size determined by the neck width(Figure schematic of device layout).
FC-40 oil with 2\% (v/v) surfactant (Pico-Surf 1, Dolomite) is the continuous phase.
FC-40 is a fluorinated oil that is preferred for biological applications.
It has a low viscosity and is denser that water.
For testing water with 1 $\mu$m fluorescent beads is used for the dispersed phase.
For transformation studies cells suspended in M9 media and inducer is used.
The droplet encapsulation procedure is preformed in a cold room to keep the cells inactivated until the desired amount of emulsion is produced(Figure (cold room set-up)).
Depending on the experiment and chip design emulsion may be stored in off-chip tubing or can be incubated on-chip.
Flow is driven with a two syringe pumps (model, brand), one for the oil syringe and one that drives both aqueous syringes.
1 m syringes (BD Biosciences) with blunted 21 gauge hypodermic needles and PTFE tubing was used.
With the syringe and tubing primed the tubing is inserted into the inlet ports.
The aqueous pump is started first at a flow rate of 50 $\mu$L/h.
The droplet generating chip is observed through a bench-top microscope.
When the aqueous channels begin making progress towards the droplet generating neck the pump with the oil syringe is started at 90 $\mu$L/h.
As the flows equilibrate droplets should begin forming.
The aqueous flow rate can then be reduced to 15 $\mu$L/h for the remainder of droplet formation.
Droplet formation can continue until the hydrophobic surface becomes fouled or debris block the neck.

\subsection{Emulsion Breaking}

After transformation cells are released from the droplets into a bulk suspension for plating.

\begin{enumerate}
	\item After transferring the emulsion to a micro centrifuge tube 200 $\mu$L CAT medium was added
	\item The tube was Spun down at 100 RCF for 30 s
	\item The separated oil was pipetted out from the bottom.
	\item 400 $\mu$L of de-emulsifying agent (Pico-Break, Dolomite) was added.
	\item The tube was spun down again for at 1000 RCF for 1 minute.
	\item Finally the cells suspended in media are pipetted off the top of the de-emulsifying agent.
\end{enumerate}

The amount of media volume in emulsion is estimated based on the packing efficiency of spheres in a sheet.
The highest volume ratio of of droplets is ~70\%.
The volume of holding chamber is 8 $\mu$L, so fully packed the volume of aqueous phase is 5.6 $\mu$L.
The volume of suspension that is recovered is assumed to be ~4 $\mu$L.
The recovered suspension is brought up to 400 $\mu$L an considered to be diluted to by 100.
