% Chapter Template

\chapter{Materials and Methods} % Main chapter title

\label{Chapter2} % Change X to a consecutive number; for referencing this chapter elsewhere, use \ref{ChapterX}

\lhead{Chapter 2. \emph{Materials and Methods}} % Change X to a consecutive number; this is for the header on each page - perhaps a shortened title

%----------------------------------------------------------------------------------------
%	SECTION 1
%----------------------------------------------------------------------------------------

\section{3D Printed Microfluidic Devices}

\subsection{Fabrication of the 3D-printed Part}
The fluidic distribution networks were designed in Autocad in 2D and then extruded to 3D with Blender.
Because the channel path length to each well varied the flow resistance in each channel was balanced by adjusting the width.
Each distribution network services six pillars that reach into the well of the plate leaving a 500 $\mu$m gap to the culture surface.
A pipe within a pipe design was used to allow a forward and return path in each pillar that also created uniform flow across the diffusion membrane.
The design was repeated to create four sets that service six wells.
Hose barbs were added to each inlet and outlet port.
The resulting STL file was printed by fineline prototyping with a Viper SLA system in WaterShed XC, an ABS or PBT-like proprietary material.

\subsection{Attachment of Membranes}
Gas permeable membranes were made by compressing 10:1 mixed and degassed PDMS between two glass plates that were spaced 100 $\mu$m apart with scotch tape.
Molds were baked at 50$^{\circ}$C on a hot plate to avoid bubble formation.
Membranes were transfered to a transparency and placed over a cutting template and cut to size for the pillar bottoms.
To attach the membranes to the 3D printed part, a small amount of PDMS was applied to the membranes and spread thin to act as a mortar.
The membranes were left in place to cure overnight.

\subsection{Oxygen Characterization}
PtOEPK (Pt(II) Octaethylporphine ketone) sensors were created by spinning thin films from a PtOEPK in polystyrene-toluene solution.
Thin film sensors were cut and fixed to the bottom of each well with PDMS.


 to each well in the network we
One central input branches to central input that equalizes the flow along each path length by varying the channel width to the proximal, intermediate, and distal wells (Fig 1). 

%-----------------------------------
%	SUBSECTION 1
%-----------------------------------
\subsection{Subsection 1}

Nunc posuere quam at lectus tristique eu ultrices augue venenatis. Vestibulum ante ipsum primis in faucibus orci luctus et ultrices posuere cubilia Curae; Aliquam erat volutpat. Vivamus sodales tortor eget quam adipiscing in vulputate ante ullamcorper. Sed eros ante, lacinia et sollicitudin et, aliquam sit amet augue. In hac habitasse platea dictumst.

%-----------------------------------
%	SUBSECTION 2
%-----------------------------------

\subsection{Subsection 2}
Morbi rutrum odio eget arcu adipiscing sodales. Aenean et purus a est pulvinar pellentesque. Cras in elit neque, quis varius elit. Phasellus fringilla, nibh eu tempus venenatis, dolor elit posuere quam, quis adipiscing urna leo nec orci. Sed nec nulla auctor odio aliquet consequat. Ut nec nulla in ante ullamcorper aliquam at sed dolor. Phasellus fermentum magna in augue gravida cursus. Cras sed pretium lorem. Pellentesque eget ornare odio. Proin accumsan, massa viverra cursus pharetra, ipsum nisi lobortis velit, a malesuada dolor lorem eu neque.

%----------------------------------------------------------------------------------------
%	SECTION 2
%----------------------------------------------------------------------------------------

\section{Main Section 2}

Sed ullamcorper quam eu nisl interdum at interdum enim egestas. Aliquam placerat justo sed lectus lobortis ut porta nisl porttitor. Vestibulum mi dolor, lacinia molestie gravida at, tempus vitae ligula. Donec eget quam sapien, in viverra eros. Donec pellentesque justo a massa fringilla non vestibulum metus vestibulum. Vestibulum in orci quis felis tempor lacinia. Vivamus ornare ultrices facilisis. Ut hendrerit volutpat vulputate. Morbi condimentum venenatis augue, id porta ipsum vulputate in. Curabitur luctus tempus justo. Vestibulum risus lectus, adipiscing nec condimentum quis, condimentum nec nisl. Aliquam dictum sagittis velit sed iaculis. Morbi tristique augue sit amet nulla pulvinar id facilisis ligula mollis. Nam elit libero, tincidunt ut aliquam at, molestie in quam. Aenean rhoncus vehicula hendrerit.
