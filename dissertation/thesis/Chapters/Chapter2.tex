% Chapter Template

\chapter{Materials and Methods} % Main chapter title

\label{Chapter2} % Change X to a consecutive number; for referencing this chapter elsewhere, use \ref{ChapterX}

\lhead{Chapter 2. \emph{Materials and Methods}} % Change X to a consecutive number; this is for the header on each page - perhaps a shortened title

%----------------------------------------------------------------------------------------
%	SECTION 1
%----------------------------------------------------------------------------------------

\section{3D Printed Microfluidic Devices}

\subsection{Fabrication of the 3D-printed Part}
The fluidic distribution networks were designed in Autocad in 2D and then extruded to 3D with Blender.
Because the channel path length to each well varied the flow resistance in each channel was balanced by adjusting the width.
Each distribution network services six pillars that reach into the well of the plate leaving a 500 $\mu$m gap to the culture surface.
A pipe within a pipe design was used to allow a forward and return path in each pillar that also created uniform flow across the diffusion membrane.
The design was repeated to create four sets that service six wells.
Hose barbs were added to each inlet and outlet port.
The resulting STL file was printed by fineline prototyping with a Viper SLA system in WaterShed XC, an ABS or PBT-like proprietary material.

\subsection{Attachment of Membranes}
Gas permeable membranes were made by compressing 10:1 mixed and degassed PDMS between two glass plates that were spaced 100 $\mu$m apart with scotch tape.
Molds were baked at 50$^{\circ}$C on a hot plate to avoid bubble formation.
Membranes were transfered to a transparency and placed over a cutting template and cut to size for the pillar bottoms.
To attach the membranes to the 3D printed part, a small amount of PDMS was applied to the membranes and spread thin to act as a mortar.
The membranes were left in place to cure overnight.

\subsection{Oxygen Characterization}
PtOEPK (Pt(II) Octaethylporphine ketone) sensors were created by spinning thin films from a PtOEPK in polystyrene-toluene solution.
Thin film sensors were cut and fixed to the bottom of each well with PDMS.


 to each well in the network we
One central input branches to central input that equalizes the flow along each path length by varying the channel width to the proximal, intermediate, and distal wells (Fig 1). 


%----------------------------------------------------------------------------------------
%	SECTION 2
%----------------------------------------------------------------------------------------

\section{Microbiology}

\subsection{\textit{Streptococcus pneumoniae} strains}

Two strains of \textit{Streptococcus pneumoniae} were produced by Dr. Morrison of the Biological Sciences Department at UIC: CP2204 and CP2215.
The strains were made to be complementary in antibiotic resistances to aid in assaying transformation events.
The CP2204 strain has inducible competence so is the designated recipient where CP2215 can not become competent so it is the natural donor of DNA.
Traits of these strains are listed in table \ref{table_strains}

\begin{table}[]
	\centering
	\caption{Summary of phenotypes of strains used for transformation studies}
	\label{table_strains}
	\begin{tabular}{llllll}
		Strain & Competence    & Rifampin  & Spectinomycin & Novobiocin & Label \\
		CP2204 & CSP inducible & Resistant & Sensitive     & Sensitive  & RFP   \\
		CP2215 & non-competent & Sensitive & Resistant     & Resistant  & GFP  
	\end{tabular}
\end{table}

\subsection{Culturing}

\textit{Streptococcus pneumoniae} strains CP2204 and CP2215 were grown separately in 12 mL of CDM with 1\% CAT medium at 37$^{\circ}$C to the desired OD (media formulas can be found in Appendix \ref{AppendixA}).
When the desired OD was reached suspended cultures were spun down at 8000 RCF for 8 minutes in a 4$^{\circ}$C centrifuge.
The supernatant was poured off and the cultures were re-suspended in M9 or a ratio of up to 20\% CDM in M9 medium depending on the application and held at 4$^{\circ}$C.

\subsection{Inducing Transformation}

Strain CP2204 was made competent with and inducer cocktail containing CSP, BSA, and \ce{CaCl2} (The specific formula can be found in Apendeix \ref{AppendixA} Table \ref{table_csp}).
The inducer cocktail is introduced to the chilled mixture of CP2204 and CP2215.
The cells can be held in at 4$^{\circ}$C in the presence of the inducer cocktail and will remain inactivated.
This suspension is then brought up to 37$^{\circ}$C to initiate cell-cell attack and transformation.
Typically these reaction suspensions are transfered to a cryogenic vials and placed in a heater block at 37$^{\circ}$C for 30 minutes.
After 30 minutes suspension are diluted ten-fold into CAT media and incubated at 37$^{\circ}$C for an hour.
This hour incubation step dilutes the inducer inactivating it and allows the generation of recombinants to emerge.

\subsection{Plating and assaying for Drug Resistance}

Suspensions were diluted in more CAT media for anticipated survivor counts.
Typically suspensions were diluted by 100,000 for single drug, 100 for two drug agar, and 10 for triple drug agar.
Agar was made from either CAT medium or THY medium by adding 4.5 g of agar per 300 mL of medium and autoclaving.
50 mm plates are filled with the following layers in order: 

\begin{enumerate}
\item 3 mL Agar
\item 1.5 mL Agar + 75 $\mu$m to 1.5 mL cell suspension quickly mixed
\item 3 mL Agar
\item 3 mL drug agar
\end{enumerate}

Layers are added at least a few minutes apart to allow them to set before the next layer is added.
Plates are incubated at in a 37$^{\circ}$C room for 48 hours before counting.
Plates are incubated upside down to slow the diffusion of drug to the cell layer.
Concentrations in the drug layer are listed in Appendix \ref{AppendixA} Table \ref{table_drug-agar}.

\section{Droplet Generation and Cell Encapsulation}

A microfluidic droplet generating device was used to encapsulate cells during transformation.
A flow focusing device with two aqueous inlets was used to keep the cells and inducer separate until right before droplet generation.

\subsection{Fabrication of Droplet Generating Device}

\textbf{Master mold fabrication.} A 2D design was made in AutoCad and exported as a DFX format.
Small features such as as the filter arrays were saved in a separate file and then subtracted via boolean operation in LinkCAD.
SU-8 2015 (MicroChem Corp.) was spun at 4500 RPM on a 100 mm silicon wafer to a thickness of 15 $\mu$.
The wafer was baked at 95$^{\circ}$C for 10 minutes shielded from light.
The wafer was loaded into the $\mu$PG 101 (Heilderberg Instruments).
The converted file from LinkCAD was imported into the $\mu$PG 101 software.
The wafer was exposed at 18 MW, 100\% intensity and 4x exposure time.
After exposure the wafer was baked at 95$^{\circ}$C for 10 minutes and then at 120$^{\circ}$C for 30 minutes.
I found that this additional bake at 120$^{\circ}$C prevented de-lamination that is typical for SU-8 heights less than 50 $\mu$m.
The uncured SU-8 was then washed away with SU-8 developer (MicroChem Corp.) on a shaker for 30 minutes.
The wafer was then rinsed with acetone followed by IPA and blown dry with nitrogen.

\textbf{Casting, assembly and treatment of device.} About 30 grams of 10:1 monomer to curing agent PDMS was mixed with a Thinky mixer for 2 minutes.
About 10 grams of mixed PDMS was poured onto the master mold and spun at 2000 RPM.
A 22 x 50 mm glass cover slip was layed on the uncured PDMS.
The mold was placed in a vacuum desiccator to pull air bubbles from under the cover slip.
The cover slip was then adjusted to be over the incubation chamber of the design and baked at 65$^{\circ}$C to set in place.
The remaining PDMS, ~20 grams, is poured on the mold and degassed.
The mold is then baked at 65$^{\circ}$C for 2 hours.
The PDMS is then carefully pealed from the mold containing the embedded cover slip.
Holes are punched with a 1 mm biopsy punch and the PDMS is placed in contact with a glass slide after 45 seconds of oxygen plasma exposure (Plasma Etch, Inc.) to permanently bond.
The completed chip is then baked at 135$^{\circ}$C for at least 1 hour to strengthen plasma bonding.
The chip is then removed and allowed to cool before treating with hydrophobic coating.
Novec 1720 (3M) is flowed through the device and then left to stand for 10 minutes.
The Novec is then flushed from the device with air and returned to a 135$^{\circ}$C hot plate to bake in the coating.
